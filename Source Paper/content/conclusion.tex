\section{conclusion}
After analysing the data we have mined and processed we aim to now answer the research questions we have described in section \ref{rqs}.

In section \ref{refactormethods} it was quite obvious that \textit{Rename Method} was by far the most used refactor method from the evaluated repositories, however the refactor methods: \textit{Rename Class} , \textit{Extract Method} and \textit{Move Class} were also used a lot. Despite the fact that these methods were used the most, they were not used consistently over all the repositories. The refactor methods: \textit{Move Attribute}, \textit{Move Method}, \textit{Inline Method} and \textit{Extract Interface} were used around the same amount of times in each project. In order to answer \textbf{RQ1}: \textit{"What type of refactorings do developers apply on testcode?"}, we could say that developers actually use almost all of the evaluated refactor methods on test code, some more than others and not all are consistently used. However, the refactor methods \textit{Extract Interface} and \textit{Rename Package} are barely used at all.

In section \ref{maintainability:improved} we have seen that there were no production classes that somehow had been affected in terms of maintainability after a significant maintainability affecting test refactoring had taken place. With this result we can say that regarding \textbf{RQ2} and \textbf{RQ2.1} the answer to the question \textit{"Does the refactoring of test code affect the maintainability of production code?"} is no, and therefore also it also does not differ per refactoring type.

Finally we have seen in section \ref{maintainability:correlation} that there seems to be a strong correlation in every metric that production classes that are classified as Very Low in terms of risk that their corresponding test class is in this same category. From this we can conclude that there is a correlation however it should be noted that this correlation does not hold for the other categories. We discussed this could be related to the fact that even more complex classes can be tested with relatively simple code, thus resulting in the higher density in the leftmost column in each of the plots shown. Therefore our answer to \textbf{RQ3}: \textit{"What is the correlation between test code maintainability and production code maintainability?"} is that maintainable\footnote{where we say maintainable is being classified as a Very Low risk class} test classes tend to also have maintainable production classes. However is it to be noted that although a large amount of test classes that we have analysed were considered maintainable, their corresponding production classes were classified as less maintainable too\footnote{by this we refer to the higher density leftmost column described in section \ref{maintainability:correlation}}.