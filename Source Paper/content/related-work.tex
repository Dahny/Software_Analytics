\section{related work}
This work is built upon the research performed by van Deursen et Al. where all kind of test code smells are identified which can be refactored by certain refactor methods \cite{van2001refactoring}. For this research a project was used (DocGen) with a high production/test file ratio. A. van Deursen et Al. classified all the different code refactors to refactor methods which were used to resolve the different test code smells. However, besides the interesting and informative results, this research lacked the overall conviction for developers to do anything more with the presented results. Where van Deursen et Al. focused on Extreme Programming and code smells, we decided to focus more on the default java projects and the maintainability of the project. 

Bois et Al. analyzed the effect of code refactors on the quality of the code \cite{du2004refactoring}. Just like A. van Deursen they searched for all the 'refactoring opportunities' (code smells) and proposed a way to refactor the code to enhance certain properties of the maintainability. They encountered many problems in classifying the analyzed code into refactor methods, however they were able to make certain guidelines for developers which, if followed, should improve the code quality of a project. Just like Bois et Al. we hope to stimulate developers to increase the quality of their project by refactoring the test code with the right refactor methods.

Increasing the quality of the code has been a driving force for analyzing code refactors by many studies. However, 'code quality' is a very broad concept which can't be formulated to one specific value. Subramanyam et Al. \cite{subramanyam2003empirical} did an empirical study on the role of object-oriented design complexity metrics in determining software defects. They used a tool (CK) which was able to analyze all the maintainability metrics of a software project. This study provided a convincing correlation between certain CK metrics and software defects.

Another approach for improving the code quality is by testing it extensively. Laitenberger et Al. \cite{laitenberger1998studying} studied the effects of code inspection and structural testing on the quality of the software. He concluded besides the importance of code inspection, that testing code is a very important requirement for detecting bugs and thereby improving the code quality.